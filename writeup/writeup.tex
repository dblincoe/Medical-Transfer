\documentclass[10pt,twocolumn,letterpaper]{article}

\usepackage{cvpr}
\usepackage{times}
\usepackage{epsfig}
\usepackage{graphicx}
\usepackage{amsmath}
\usepackage{amssymb}

% Include other packages here, before hyperref.

% If you comment hyperref and then uncomment it, you should delete
% egpaper.aux before re-running latex.  (Or just hit 'q' on the first latex
% run, let it finish, and you should be clear).
\usepackage[breaklinks=true,bookmarks=false]{hyperref}

\cvprfinalcopy % *** Uncomment this line for the final submission

\def\cvprPaperID{****} % *** Enter the CVPR Paper ID here
\def\httilde{\mbox{\tt\raisebox{-.5ex}{\symbol{126}}}}

% Pages are numbered in submission mode, and unnumbered in camera-ready
%\ifcvprfinal\pagestyle{empty}\fi
\begin{document}

%%%%%%%%% TITLE
\title{A Transfer Learning Approach to Aggregated DICOM Lung Tumor Classification}

\author{Kennan LeJeune\\
   {\tt\small kennan@case.edu}
   % For a paper whose authors are all at the same institution,
   % omit the following lines up until the closing ``}''.
   % Additional authors and addresses can be added with ``\and'',
   % just like the second author.
   % To save space, use either the email address or home page, not both
   \and
   David Blincoe\\
   {\tt\small drb133@case.edu}
   \and
   Sam Jenkins\\
   {\tt\small soj3@case.edu}
   \and
   Chris Toomey\\
   {\tt\small ctt16@case.edu}
   \and
   Arthur Xin\\
   {\tt\small sxx132@case.edu}\\
   \and
   {Case Western Reserve University}\\
   {\textit{Department of Computer and Data Sciences}}
}
\maketitle
%\thispagestyle{empty}

%%%%%%%%% ABSTRACT
\begin{abstract}
   Lung tumor identification and classification is a challenging task which typically requires a trained medical
   professional to choose the best slices of a scan and accurately classify the chosen slices. With medical data privacy restrictions
   and regulations, it is difficult to collect sufficient data to construct a typical Convolutional Neural Network to
   choose and classify DICOM slices. We propose an inductive transfer learning approach which applies hidden layer
   image representations from a residual neural network to our lung nodule classifier to classify each slice and provide
   recommendations to a user as to what slices may contain possible nodules.
\end{abstract}

\section{Introduction} \label{sec:intro}

   Traditionally, Machine Learning problems rely on the assumption that the training data and future data domains
   are in the same feature space and have the same distribution (Pan, Yang). However, a large number of real world
   applications perform poorly when these assumptions are not met. A common instance of is a classification task on
   a restricted domain of interest (e.g. classifying malignant brain tumors) with minimal or unlabeled training data,
   but we have sufficient training data in another domain of interest (e.g. classifying malignant lung carcinomas).
   Transfer Learning aims to solve this problem by adapting the knowledge acquired from a training task or
   training domain for use in a related domain or task. Intuitively, we can apply a solution to a problem to a
   different but related problem in a human-like matter.

   \subsection{Formal Transfer Learning Definition} \label{sec:intro-def}
      Given source and target domains $D_S, D_T$ and learning tasks $T_S$ on the source domain and $T_T$ on the target
      domain, where we aim to improve the learning of a target predictive function $f_T(\dot)\in D_T$ using
      the knowledge in $D_S$ or $T_S$, where $D_S\neq D_T$ or $T_S\neq T_T$. Note that in the case where both domains
      and tasks are equivalent, this is analogous to a standard machine learning problem. We can characterize the
      nature of nearly all Transfer Learning problems by considering three primary cases:

   \subsection{Types of Transfer Learning} \label{sec:intro-types}
      1. $D_S = D_T, T_S\neq T_T$ (Inductive Transfer, \ref{sec:intro-types-inductive})

      2. $D_S\neq D_T, T_S = T_T$ (Transductive Transfer, \ref{sec:intro-types-transductive})
    
      3. $T_S\neq T_T$ and $D_S, D_T$ are unlabeled (Unsupervised Transfer, \ref{sec:intro-types-unsupervised})
    
      \subsubsection{Inductive Transfer Learning} \label{sec:intro-types-inductive}
    
         In Inductive Transfer Learning, the domain of the source and target tasks is the same. In this case, labeled data
         from the target task must be available in order to induce the predictive function that we want the classifier to learn.
         There are two situations in inductive transfer learning: one where labeled source data is available, and one where it
         is not. The first scenario is similar to something known as self-taught learning, in which the classifier hopes to
         learn basic patterns from random unlabeled data. The second scenario is similar to multi task learning, where the
         classifier attempts to learn several classification tasks at the same time, except in this case we only care about
         the performance on a single target task, and are hoping to use knowledge learned from the other tasks in order to
         improve performance on said target task.
    
      \subsubsection{Transductive Transfer Learning} \label{sec:intro-types-transductive}
         Transductive Transfer Learning focuses on an area in which the source and target tasks are the same, but the domains
         differ. The data used in the target domain is unlabeled, whereas there is an abundance of labeled data in the source
         domain. Transductive transfer learning can be broken down even further, into two specific scenarios: the first, in
         which the target and source domains have a different feature space; and the second, where the feature space is the
         same, but the input data have different marginal probability distributions.
    
      \subsubsection{Unsupervised Transfer Learning} \label{sec:intro-types-unsupervised}
         Unsupervised Transfer Learning focuses on a setting where there is no labeled data for either the source and target
         domains, and the target task is different from the source task. This situation is common in areas such as clustering
         and dimensionality reduction.

 \subsection{Classification Problem} \label{sec:intro-class}

​	Lung cancer is one of the most dangerous and lethal diseases worldwide. A patient’s probability of survival decreases the longer it takes to identify a cancerous tumor. Thus, successful lung cancer screenings can be used to save countless lives, making them extremely valuable. 



%-------------------------------------------------------------------------

\section{Related Works} \label{sec:works}

\section{Datasets} \label{sec:data}
   For this research we needed to obtain two separate datasets to fit two
   distinct roles. ImageNet, \ref{sec:data-imagenet}, was needed to train
   the classifier on general image data to fit the convolution segment of
   the network to recognize underlying image Structure. This is our source
   domain, $D_s$.

   LIDC-IDRI, \ref{sec:data-lidc}, is used to train the output segment of the
   the network. This is our target domain, $D_t$.

   \subsection{ImageNet Dataset} \label{sec:data-imagenet}

   \subsection{LIDC-IDRI Dataset} \label{sec:data-lidc}
      The Lung Image Database Consortium image collection (LIDC-IDRI) dataset
      is a very popular cancer classification dataset that focuses on tumors located
      in the lungs. The lung scans are CT images of the upper torso. The entire dataset
      consists of 1018 cases that each contain thoracic radiologists annotations of tumor segments
      These tumors annotations each contain 9 different descriptors such as malignancy,
      calcification, and lobulation. The descriptor this paper is interested in is the
      malignancy of each nodule.

      The malignancy is rated on a 1-5 scale. 1 being 'Highly Unlikely' of malignancy and
      5 being 'Highly Suspicious' of malignancy. Using these ratings each slice in a chest CT
      was rated as either malignant, benign, or non-nodule. A slice was considered non-nodule
      if there was no nodule annotations found within the slice. To split nodules into malignant
      and benign labels, the annotations performed on a specific node were averaged and for malignancy
      values $\ge 3$, the node was considered malignant and for malignancy values $< 3$, the node was
      considered benign, based upon the 4 radiologists predictions.
    
      \subsubsection{PyLIDC} \label{sec:data-lidc-pylidc}
         To assist in the extraction of data from the DICOM image files, a python library was utilized
         to read to XML files which contained the annotation information for each nodule. \cite{Hancock2018}
    
      \subsubsection{Processing LIDC-IDRI Data} \label{sec:data-lidc-processing}
         DICOM files, (Digital Imaging and Communications in Medicine), are the standard method for transferring
         and communicating image data. The structures of these files are extremely robust and offer many access in
         the form of 'Tags'. In the case of LIDC-IDRI, the dataset is composed entirely of CT images which must be
         processed by first transforming the image data along the HU (Hounsfield scale) given the transformation
         coefficients in the DICOM.
    
         The vertical slice size must also be taken into account because CT scans can be ordered in a variety of 
         ranging resolutions from $(<0.1\text{mm to } >3 \text{mm})$. A scale of 1 mm per slice was chosen and the
         pixel data was transformed.

\section{Network Structure} \label{sec:struct}

   \subsection{Residual Neural Network} \label{sec:struct-cnn}
        The vanishing gradient has long been a problem when constructing neural networks with large architectures. Essentially, the back-propagating the gradient to earlier layers makes the gradient tend towards zero, so the performance can degrade in earlier layers. This can result in worse performance for deeper models than shallower ones, since earlier layers will perform worse for the deeper model. Residual neural networks attempt to solve this problem, by using "shortcuts" that jump layers in order to make sure the gradient does not become infinitesimally smaller. The image below illustrates this concept. The input, $X$, is both passed to the next layer and skips ahead to the layer after, in order to add the effects of the input and the activation function and avoid the vanishing gradient problem. 

![Resnet](Resnet.png)

​	For our project, we are using the ResNet-50 pretrained model for Keras. The architecture of the model can be seen in the image below. 

![Keras Resnet-50](Keras Resnet-50.png)

As a whole, the mode has 5 stages, followed by a pooling and flattening step to output the data. The first stage puts the data through a convulutional layer that then runs ReLU and max pooling, and then the model consists of four stages convolutional blocks and identity blocks, with varying number of identity blocks. The convolutional blocks runs the input through two iterations of convolutional layers, followed by ReLU, and a final convolutional layer, and adding to this the input run through a convolutional layer, before running a final iteration of ReLU. The identity block does the same, but simply adds back the input in the end rather than a convolutional layer run on the input. The difference in the two is the shape of the input and output: if their dimensions line up, then an identity block is sufficient, otherwise the input must be re-shaped to match the output. 

   \subsection{Study Aggregation} \label{sec:struct-aggr}

\section{Experiments} \label{sec:experiments}

\section{Results} \label{sec:results}

\section{Conclusion} \label{sec:conclusion}

{\small
\bibliographystyle{ieee}
\bibliography{egbib}
}

\end{document}
