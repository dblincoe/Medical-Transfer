\documentclass[10pt,twocolumn,letterpaper]{article}

\usepackage{cvpr}
\usepackage{times}
\usepackage{epsfig}
\usepackage{graphicx}
\usepackage{amsmath}
\usepackage{amssymb}

% Include other packages here, before hyperref.

% If you comment hyperref and then uncomment it, you should delete
% egpaper.aux before re-running latex.  (Or just hit 'q' on the first latex
% run, let it finish, and you should be clear).
\usepackage[breaklinks=true,bookmarks=false]{hyperref}

\cvprfinalcopy % *** Uncomment this line for the final submission

\def\cvprPaperID{****} % *** Enter the CVPR Paper ID here
\def\httilde{\mbox{\tt\raisebox{-.5ex}{\symbol{126}}}}

% Pages are numbered in submission mode, and unnumbered in camera-ready
%\ifcvprfinal\pagestyle{empty}\fi
\begin{document}

%%%%%%%%% TITLE
\title{A Transfer Learning Approach to Aggregated DICOM Lung Tumor Classification}

\author{Kennan LeJeune\\
   {\tt\small kennan@case.edu}
   % For a paper whose authors are all at the same institution,
   % omit the following lines up until the closing ``}''.
   % Additional authors and addresses can be added with ``\and'',
   % just like the second author.
   % To save space, use either the email address or home page, not both
   \and
   David Blincoe\\
   {\tt\small drb133@case.edu}
   \and
   Sam Jenkins\\
   {\tt\small soj3@case.edu}
   \and
   Chris Toomey\\
   {\tt\small ctt16@case.edu}
   \and
   Arthur Xin\\
   {\tt\small sxx132@case.edu}\\
   \and
   {Case Western Reserve University}\\
   {\textit{Department of Computer and Data Sciences}}
}
\maketitle
%\thispagestyle{empty}

%%%%%%%%% ABSTRACT
\begin{abstract}
   
\end{abstract}

\section{Introduction} \label{intro}

Transfer Learning is a technique that offers a unique approach to problems
that require

   \subsection{Problem Definition} \label{intro-def}
%-------------------------------------------------------------------------

\section{Related Works} \label{works}

\section{Datasets} \label{data}
   For this research we needed to obtain two separate datasets to fit two
   distinct roles. ImageNet, \ref{data-imagenet}, was needed to train
   the classifier on general image data to fit the convolution segment of
   the network to recognize underlying image Structure. This is our source
   domain, $D_s$.

   LIDC-IDRI, \ref{data-lidc}, is used to train the output segment of the
   the network. This is our target domain, $D_t$.

   \subsection{ImageNet Dataset} \label{data-imagenet}
   
   
   \subsection{LIDC-IDRI Dataset} \label{data-lidc}
      The Lung Image Database Consortium image collection (LIDC-IDRI) dataset
      is a very popular cancer classification dataset that focuses on tumors located
      in the lungs. The lung scans are CT images of the upper torso. The entire dataset
      consists of 1018 cases that each contain thoracic radiologists annotations of tumor segments
      These tumors annotations each contain 9 different descriptors such as malignancy,
      calcification, and lobulation. The descriptor this paper is interested in is the
      malignancy of each nodule.

      The malignancy is rated on a 1-5 scale. 1 being 'Highly Unlikely' of malignancy and 
      5 being 'Highly Suspicious' of malignancy. Using these ratings each slice in a chest CT
      was rated as either malignant, benign, or non-nodule. A slice was considered non-nodule
      if there was no nodule annotations found within the slice. To split nodules into malignant
      and benign labels, the annotations performed on a specific node were averaged and for malignancy
      values $\ge 3$, the node was considered malignant and for malignancy values $< 3$, the node was
      considered benign, based upon the 4 radiologists predictions.

      \subsubsection{PyLIDC} \label{data-lidc-pylidc}
         To assist in the extraction of data from the DICOM image files, a python library was utilized
         to read to XML files which contained the annotation information for each nodule. \cite{Hancock2018}

         \section{Network Structure} \label{struct}
         
   \subsection{Convolutional Neural Network} \label{struct-cnn}

   \subsection{Aggregation SVM} \label{struct-svm}

\section{Experiments} \label{experiments}

\section{Results} \label{results}

\section{Conclusion} \label{conclusion}

{\small
\bibliographystyle{ieee}
\bibliography{egbib}
}

\end{document}
